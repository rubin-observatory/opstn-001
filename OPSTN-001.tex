

\documentclass[OPS,authoryear,toc]{lsstdoc}
% lsstdoc documentation: https://lsst-texmf.lsst.io/lsstdoc.html

% Package imports go here.

\usepackage[nonumberlist,nogroupskip,toc,numberedsection=autolabel]{glossaries}
\usepackage{environ}
\usepackage{enumitem}

\setmonofont[AutoFakeSlant]{Inconsolata}




% Local commands go here.

% DO NOT EDIT - generated by /Users/womullan/LSSTgit/lsst-texmf/bin/generateAcronyms.py from https://lsst-texmf.lsst.io/.
\newacronym {AD} {AD} {Associate Director}
\newacronym {AMCL} {AMCL} {Aura Management Council for LSST}
\newacronym {AWS} {AWS} {Amazon Web Services, one of the largest cloud computing providers.}
\newacronym {CI} {CI} {\gls{cyberinfrastructure}}
\newacronym {DAQ} {DAQ} {Data Acquisition System}
\newacronym {DBB} {DBB} {Data Back Bone}
\newacronym {DMTN} {DMTN} {DM Technical Note}
\newacronym {DOE} {DOE} {Department Of Energy}
\newacronym {FTE} {FTE} {Full Time Equivalent}
\newacronym {ITC} {ITC} {Information Technology Center}
\newacronym {LDF} {LDF} {LSST Data Facility}
\newacronym {LSST} {LSST} {Large Synoptic Survey Telescope}
\newacronym {NCOA} {NCOA} {National Center for Optical-Infrared Astronomy}
\newacronym {NCSA} {NCSA} {National Center for Supercomputing Applications}
\newacronym {OPS} {OPS} {Operations}
\newacronym {OPSTN} {OPSTN} {Operations Technical Note, document identifier handle}
\newacronym {QA} {QA} {Quality Assurance}
\newglossaryentry {Qserv} {name={Qserv}, description={Proprietary LSST Database system}}
\newglossaryentry {S3} {name={S3}, description={Structured, imperative high level computer programming language, used as implementation language for the Virtual Machine Environment (\gls{VME}) operating system}}
\newacronym {SDQA} {SDQA} {Science Data Quality Assurance}
\newacronym {SLAC} {SLAC} {No longer an acronym; formerly Stanford Linear Accelerator Center}
\newacronym {SRE} {SRE} {Site Reliability Engineering}
\newglossaryentry {cloud} {name={cloud}, description={A visible mass of condensed water vapor floating in the atmosphere, typically high above the ground or in interstellar space acting as the birthplace for stars.  Also a way of computing (on other peoples computers leveraging their services and availability).}}
\newglossaryentry {cyberinfrastructure} {name={cyberinfrastructure}, description={Sometimes denoted CI, A term first used by the US National Science Foundation (\gls{NSF}), and it typically is used to refer to information technology systems that provide particularly powerful and advanced capabilities.}}
\newglossaryentry {software} {name={software}, description={The programs and other operating information used by a computer.}}

\makeglossaries

% To add a short-form title:
% \title[Short title]{Title}
\title{Alternative science operations approach for  LSST }

% Optional subtitle
% \setDocSubtitle{A subtitle}

\author{%
William O'Mullane
}

\setDocRef{OPSTN-001}

\date{\today}

% Optional: name of the document's curator
% \setDocCurator{The Curator of this Document}

\setDocAbstract{%
After the Kavli workshop Petabytes to Science we have a suggested framework for data processing and disemination. This note cosiders how LSST might fit in such a scheme.
}

% Change history defined here.
% Order: oldest first.
% Fields: VERSION, DATE, DESCRIPTION, OWNER NAME.
% See LPM-51 for version number policy.
\setDocChangeRecord{%
  \addtohist{1}{YYYY-MM-DD}{Unreleased.}{William O'Mullane}
}

\begin{document}

% Create the title page.
% Table of contents is added automatically with the "toc" class option.

\mkshorttitle
%switch to \maketitle if you wan the title page and toc


% ADD CONTENT HERE ... a file per section can be good for editing
\section{Introduction} \label{sec:intro}

We are currently experimenting with Google and Amazon services for science platform and processing.
These services are priced to deliver compute and storage - our current model at the \gls{LDF} is also service oriented but
is not proceed in the same manner making comparisons difficult. It has been difficult to get an alternative cost model
together. An initial approach to a \gls{cloud} costing was outlined in \citeds{DMTN-072}, this approach was to try to cost the hardware and compare to \gls{cloud} pricing.

In this document a radical restructuring of \gls{LSST} operations is explored - a technology stack underpinned by commodity services which could be provided by commercial providers or computing centers. Here we look first at how we would run something like this -we can then leave one free variable which is the cost of the underlying compute and storage services. This will both help to
sanity check the \gls{LDF} costing and potentially allow us to have a ball park for assessing commodity provider offers.



\section{Plugable service oriented architecture} \label{sec:arc}

In the Kavli workshop in Vegas (Feb 2019) we took a long term view to astronomy archives and data processing.
We suggested a layered service model as depicted in \figref{fig:CI}\footnote{The full document is here \url{https://petabytestoscience.github.io/PetaBytes-2019-04-26.pdf}}.
Our requirements are no longer unique and we have access to a wealth of open source \gls{software}, commodity hardware, and managed \gls{cloud} services (offered by commercial providers and federally-funded institutions) that are well positioned to meet the needs of LSST \cite{2019AAS...23345706M, 2019AAS...23324505B}.


\begin{figure}
    \centering
    \includegraphics[width=1.\textwidth]{images/CI-Layers}
    \caption{An example a \gls{cyberinfrastructure} built on an Infrastructure as Code design model. Note that while this example does not have astronomy-specific tooling, our recommendations highlight the importance of developing astro-specific layers that are fully accessible to scientists in both  the application  and the graphical interface layers. \label{fig:CI}}
\end{figure}


We took \figref{fig:CI} and made a more \gls{LSST} oriented version in \figref{fig:CI-LSST}. This is pretty close to how we are currently but we do not treat the compute and storage as pure services.

\begin{figure}
    \centering
    \includegraphics[width=1.\textwidth]{images/CI-LSST}
    \caption{An example \gls{LSST}  \gls{cyberinfrastructure} built analogous to the \gls{CI} model shown in \figref{fig:CI}.}
% Original https://docs.google.com/presentation/d/16w5WVe-_xNLXWudNKqJu9IEKGDQ8oIB0eDiqWP_SUnE/edit?usp=sharing
    \label{fig:CI-LSST}
\end{figure}\







\section{Science operations}\label{sec:sciops}

The role of science operations within LSST is to deliver LSST's science products: the science images, the alert stream, the annual data releases, the science software, and the Science Platform. In the current ops proposal not all groups required to do this are under control of Science operations.

\figref{fig:org} gives a slightly augmented view of the science operations department.

\begin{figure}
\includegraphics[width=0.9\textwidth]{figures/OrgOpsChart}
\caption{Possible configuration of Science Operations Department for operations of LSST \label{fig:org}}
% original https://docs.google.com/presentation/d/1wIN6Dj_rPn8TASBUkAm6_-Yh255Gz9VbwFi61t4LCFs/edit#slide=id.g55f7c0247e_0_2
\end{figure}


\subsection{Other implied changes to the current operations proposal}
Notably missing from \figref{fig:org} is QA. Currently QA is spread across  three
 departments - the suggestion here is to place all QA activities under the survey performance department. Consolidation
of the QA activities in one department may allow for some personnel saving.

The data release team in science operations would require a verification scientist (this may be 0.5FTE) while the SDQA and Semantic scientists may move to QA in survey science.

All data facility work, be it with a partner or in commercial cloud should be firmly under science operations - hence there is no LDF department and no associate director for LDF.\footnote{This is  in line with AMCL recommendations}




\section{Conclusion}\label{sec:conclusion}
 A restructuring of operations would  give more transparent cost, allow for a better comparison to commodity pricing for many services and would
yield considerable savings.





\appendix
% Include all the relevant bib files.
% https://lsst-texmf.lsst.io/lsstdoc.html#bibliographies
\section{References} \label{sec:bib}
\bibliography{lsst,lsst-dm,refs_ads,refs,books}

%Make sure lsst-texmf/bin/generateAcronyms.py is in your path
\section{Acronyms and glossary items}\label{sec:acronyms}
%having problems for now
%\printnoidxglossaries
\printglossaries
\addtocounter{table}{-1}
\begin{longtable}{|l|p{0.8\textwidth}|}\hline
\textbf{Acronym} & \textbf{Description}  \\\hline

AMCL & Aura Management Council for LSST \\\hline
AWS & Amazon Web Services, one of the largest cloud computing providers. \\\hline
CI & \gls{cyberinfrastructure} \\\hline
DM & Data Management \\\hline
DMTN & DM Technical Note \\\hline
FTE & Full Time Equivalent \\\hline
ITC & Information Technology Center \\\hline
LDF & LSST Data Facility \\\hline
LSST & Large Synoptic Survey Telescope \\\hline
NCOA & National Center for Optical-Infrared Astronomy \\\hline
NCSA & National Center for Supercomputing Applications \\\hline
NSF & National Science Foundation \\\hline
OPS & Operations \\\hline
QA & Quality Assurance \\\hline
SDQA & Science Data Quality Assurance \\\hline
US & United States \\\hline
cloud & A visible mass of condensed water vapor floating in the atmosphere, typically high above the ground or in interstellar space acting as the birthplace for stars.  Also a way of computing (on other peoples computers leveraging their services and availability). \\\hline
cyberinfrastructure & Sometimes denoted CI, A term first used by the US National Science Foundation (\gls{NSF}), and it typically is used to refer to information technology systems that provide particularly powerful and advanced capabilities. \\\hline
software & The programs and other operating information used by a computer. \\\hline
\end{longtable}

\end{document}

\subsection{Service budget}

If we assume we do not reduce the operations budget then the total cost of the \gls{LDF} services is
the difference in \gls{FTE} and the non labor costs for computer purchases.
This is calculated in \tabref{tab:Services}. One must also bear in mind that the data volume etc. increases each year,
initial costs could be a little lower with final costs being a little higher. Service prices though are dropping every year so do we add significant data each year our costs should not in crease by the same fraction.

 \begin{longtable} { |p{0.3\textwidth}  |r  |r |} 
\caption{Estimate of service budget/cost using \gls{FTE} and non-labor costs from the proposal.  \label{tab:Services}}\\ 
\hline 
\textbf{}&\textbf{FTE}&\textbf{Cost K\$} \\ \hline
{Annual labour diff from \tabref{tab:FTE}}&{17.35}&{\$4,338} \\ \hline
{Non labour hardware (average of all years - NOT CORRECT NUMVER)}&{}&{\$8,000} \\ \hline
\textbf{Total}&\textbf{}&\textbf{\$12,338} \\ \hline
\end{longtable}



We would still require some hardware on the mountain and the base in Chile where we would potentially still keep a copy of the raw data.

The  \gls{LSST} data volumes are in  \tabref{tab:sizes}.

\input{images/sizetab}



The compute estimates are a little more difficult to extract in \tabref{tab:Inputs} from \citeds{DMTN-072}
an estimate is made in terms of FLOPs.

\tiny \begin{longtable} { |p{0.22\textwidth}  |r  |r  |r  |r  |r  |r |} 
\caption{Various inputs for deriving costs \label{tab:Inputs}}\\ 
\hline 
\textbf{Year}&\textbf{2017}&\textbf{2018}&\textbf{2019}&\textbf{2020}&\textbf{2021}&\textbf{2022} \\ \hline
{FLOPs Needed Total (no Alerts)}&{9.48261E+19}&{1.00E+19}&{1.00E+19}&{9.48261E+19}&{1.00E+19}&{4.74131E+20} \\ \hline
{Time to Process days}&{252.0}&{365.0}&{365.0}&{252.0}&{365.0}&{252.0} \\ \hline
{Time to Process seconds}&{21772800.0}&{31536000.0}&{31536000.0}&{21772800.0}&{31536000.0}&{21772800.0} \\ \hline
{Instantaneous GFLOP/ s}&{4355.255691}&{3.17E+02}&{3.17E+02}&{4355.255691}&{3.17E+02}&{21776.27846} \\ \hline
{Instantaneous GFLOP/ s (inc Alerts)}&{4355.255691}&{3.17E+02}&{3.17E+02}&{30025.25569}&{2.60E+04}&{21776.27846} \\ \hline
{Disk Space TB}&{1000}&{1000}&{1000}&{10000}&{20000}&{30000} \\ \hline
{I/ O for year TB}&{10}&{100}&{3000}&{30000}&{60000}&{90000} \\ \hline
{Base numbers }&{Ecyc }&{FLOP}&{GFLOP}&&& \\ \hline
{LDM-138 DR1,2 Data Rel sheet row 1}&{155.17}&{4.26718E+20}&{426717500000}&&& \\ \hline
{LDM-138 DR3 Data Rel sheet row 2}&{348.76}&{9.5909E+20}&{959090000000}&&& \\ \hline
{LDM-138 Alert Instananeous}&{0.00023434}&{25670000000000}&{25670}&&& \\ \hline
{Alert Total, assuming 275k visits/ year}&{64.4435}&{1.7722E+20}&{177219625000}&&& \\ \hline
\textbf{Total Yr1 (inc DAC)}&\textbf{}&\textbf{4.74131E+20}&\textbf{474130555556}&&& \\ \hline
{}&{Optimistic}&{Pessimistic}&&&& \\ \hline
{Moore Factor Proc}&{0.7}&{0.9}&&&& \\ \hline
{Kryder Factor Disk}&{0.8}&{0.9}&&&& \\ \hline
\end{longtable} \normalsize


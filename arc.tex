
\section{Plugable service oriented architecture} \label{sec:arc}

In the Kavli workshop in Vegas (Feb 2019) we took a long term view to astronomy archives and data processing.
We suggested a layered service model as depicted in \figref{fig:CI}\footnote{The full document is here \url{https://petabytestoscience.github.io/PetaBytes-2019-04-26.pdf}}.
Our requirements are no longer unique and we have access to a wealth of open source \gls{software}, commodity hardware, and managed \gls{cloud} services (offered by commercial providers and federally-funded institutions) that are well positioned to meet the needs of \gls{LSST} \cite{2019AAS...23345706M, 2019AAS...23324505B}.


\begin{figure}
    \centering
    \includegraphics[width=1.\textwidth]{images/CI-Layers}
    \caption{An example a \gls{cyberinfrastructure} built on an Infrastructure as Code design model. Note that while this example does not have astronomy-specific tooling, our recommendations highlight the importance of developing astro-specific layers that are fully accessible to scientists in both  the application  and the graphical interface layers. \label{fig:CI}}
\end{figure}


We took \figref{fig:CI} and made a more \gls{LSST} oriented version in \figref{fig:CI-LSST}. This is pretty close to how we are currently but we do not treat the compute and storage as pure services.

\begin{figure}
    \centering
    \includegraphics[width=1.\textwidth]{images/CI-LSST}
    \caption{An example \gls{LSST}  \gls{cyberinfrastructure} built analogous to the \gls{CI} model shown in \figref{fig:CI}.}
% Original https://docs.google.com/presentation/d/16w5WVe-_xNLXWudNKqJu9IEKGDQ8oIB0eDiqWP_SUnE/edit?usp=sharing
    \label{fig:CI-LSST}
\end{figure}\

In important part of following this architecture is the ability to choose best in class components. In the science platform for example we have probably the best notebook implementation but we could possibly pick up a better portal. In processing \gls{NCSA} insist on  a shared noting approach - a move to an \gls{Object Storage} could profoundly change that. It may raise other questions though on replication and redundancy. Hence getting to operations in this model will require some rethinking in construction - construction is a big ship which is already steaming ahead so a change in course will take some effort. It is absolutely worth pursuing though.



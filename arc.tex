
\section{Plugable service oriented architecture} \label{sec:arc}

At the Kavli workshop held in Las Vegas (Feb 2019 \cite{2019arXiv190505116B}) we took a long term view of astronomy archives and data processing.
We suggested a layered service model as depicted in \figref{fig:CI}\footnote{The full document is here \url{https://petabytestoscience.github.io/PetaBytes-2019-04-26.pdf}}.
Our astronomy requirements are no longer unique and we have access to a wealth of open source \gls{software}, commodity hardware, and managed \gls{cloud} services (offered by commercial providers and federally-funded institutions) that are well positioned to meet the needs of \gls{LSST} \cite{2019AAS...23345706M, 2019AAS...23324505B}.

\begin{figure}
    \centering
    \includegraphics[width=1.\textwidth]{images/CI-Layers}
    \caption{An example of a \gls{cyberinfrastructure} built on an Infrastructure as Code design model. Note that while this example does not have astronomy-specific tooling, our recommendations highlight the importance of developing astro-specific layers that are fully accessible to scientists in both  the application  and the graphical interface layers. \label{fig:CI}}
\end{figure}


We took \figref{fig:CI} and made a more \gls{LSST} oriented version in \figref{fig:CI-LSST}. This is pretty close to how we are currently planning to operate \gls{LSST}, but we do not yet treat the compute and storage as pure services.

\begin{figure}
    \centering
    \includegraphics[width=1.\textwidth]{images/CI-LSST}
    \caption{An example \gls{LSST}  \gls{cyberinfrastructure} built analogous to the \gls{CI} model shown in \figref{fig:CI}.}
% Original https://docs.google.com/presentation/d/16w5WVe-_xNLXWudNKqJu9IEKGDQ8oIB0eDiqWP_SUnE/edit?usp=sharing
    \label{fig:CI-LSST}
\end{figure}\

An important part of following this architecture design is the ability to choose best in class components. In the science platform, for example, we have probably the best notebook implementation, but we could possibly pick up a better portal. In processing \gls{NCSA} insist on a shared nothing approach - a move to an \gls{Object Storage} could profoundly change that. It may raise other questions though on replication and redundancy. Then the shared nothing approach brings its own problems for deployment and \gls{QA}.

A more service oriented approach should allow us to move between service providers to use the best in class for our underlying services as well.  A clear model, understood by many, will make \gls{QA} an easier task as well.

Getting to operations in this model will require some rethinking in construction -- construction is a big ship which is already steaming ahead, so a change in course will take some effort. It is absolutely worth pursuing though.


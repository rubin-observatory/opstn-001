
\section{Science operations}\label{sec:sciops}

The role of science operations within LSST is to deliver LSST's science products: the science images, the alert stream, the annual data releases, the science software, and the Science Platform. In the current ops proposal not all groups required to do this are under control of Science operations.

\figref{fig:org} gives a slightly augmented view of the science operations department.

\begin{figure}
\includegraphics[width=0.9\textwidth]{figures/OrgOpsChart}
\caption{Possible configuration of Science Operations Department for operations of LSST \label{fig:org}}
% original https://docs.google.com/presentation/d/1wIN6Dj_rPn8TASBUkAm6_-Yh255Gz9VbwFi61t4LCFs/edit#slide=id.g55f7c0247e_0_2
\end{figure}


\subsection{Other implied changes to the current operations proposal}
Notably missing from \figref{fig:org} is QA. Currently QA is spread across  three
 departments - the suggestion here is to place all QA activities under the survey performance department. Consolidation
of the QA activities in one department may allow for some personnel saving.

The data release team in science operations would require a verification scientist (this may be 0.5FTE) while the SDQA and Semantic scientists may move to QA in survey science.

All data facility work, be it with a partner or in commercial cloud should be firmly under science operations - hence there is no LDF department and no associate director for LDF.\footnote{This is  in line with AMCL recommendations}


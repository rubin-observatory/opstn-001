\newpage
\section{Data Production Department }\label{sec:sciops} \label{sec:dataprod}

The role of data production within \gls{LSST} is to deliver \gls{LSST}'s science products: the science images, the alert stream, the annual data releases, the science \gls{software}, and the Science Platform. In the current ops proposal not all groups required to do this are under control of the  science operations \gls{AD}.

\figref{fig:opsorg} gives a view of the Data Production teams which combine some of the old science operations  and \gls{LDF} departments. This is far more analogous to Data Management moving into operations than in the current proposal and would make for a smoother transition.

\begin{figure}
\begin{center}
\includegraphics[width=0.9\textwidth,trim=0cm 10cm 0 0, clip]{figures/SciOpsOrg}
\caption{Possible configuration of Science Operations Department for operations of \gls{LSST} \label{fig:sciopsorg}}
\end{center}
% original https://docs.google.com/presentation/d/1wIN6Dj_rPn8TASBUkAm6_-Yh255Gz9VbwFi61t4LCFs/edit#slide=id.g55f7c0247e_0_2
\end{figure}

The \gls{FTE} counts are estimated in \tabref{tab:FTE}, which also gives the team sizes form the operations proposal for comparison.
A brief description of the teams is given in \secref{sec:teams}.
 \begin{longtable} { |p{0.3\textwidth}  |r  |r  |r  |r |} 
\caption{Size of the various teams in data production department with size from the proposal in the fourth column - a zero imples the team did not exist inthe proposal. \label{tab:FTE}}\\ 
\hline 
\textbf{Team}&\textbf{FTE 2023}&\textbf{FTE 2026}&\textbf{Prop}&\textbf{Note} \\ \hline
{Management (\gls{AD})}&{1.5}&{1.5}&{1}& \\ \hline
{Observatory Science }&{5.5}&{5.5}&{5.5}&{System performance?} \\ \hline
{Science Platform}&{6}&{5}&{6}& \\ \hline
{Science Algos and Pipelines}&{22}&{16}&{16}&{QA  to system performance?} \\ \hline
{Middleware}&{6.5}&{6.5}&{0}& \\ \hline
{Infrastructure}&{7}&{7}&{0}&{ } \\ \hline
{Verification/ Operations}&{4}&{4}&{0}& \\ \hline
{LDF Management (\gls{AD}) }&{0}&{0}&{3}&{In infrastruture} \\ \hline
{LDF Scientific Prod. Services}&{0}&{0}&{6.75}&{In Verification/ operations} \\ \hline
{LDF \gls{ITC} Security}&{0}&{0}&{8.5}&{Some in infrastructure/ middleware } \\ \hline
{LDF Production Services}&{0}&{0}&{4.3}&{Some in infrastructure} \\ \hline
{LDF \gls{ITC} and Facilities}&{0}&{0}&{12.25}&{Should be in services charges} \\ \hline
\textbf{Total}&\textbf{52.5}&\textbf{45.5}&\textbf{63.3}& \\ \hline
\end{longtable}


{\bf Note:} The numbers in \tabref{tab:FTE} assume we choose the people in the roles based on experience and effectiveness - in the current plan there are a number of \gls{DOE} provided FTEs where the reason in some case seems to be availability rather than suitability. In the submitted proposal there was probably a certain number of duplicated roles (certainly in \gls{LDF}) to cover this. We must consider this aspect carefully.

\newpage
\subsection{Teams }\label{sec:teams}
\figref{fig:sciopsorg} introduces several teams some of which were not in the original ops proposal. A little detail is given here about each.

\subsubsection{Observatory Science}
As in the ops proposal the primary responsibility of this team is to understand the end-to-end impact of the Observatory hardware and environment on the science images and to work with the Observatory \gls{Operations} department to ensure that the image quality meets requirements.

\textbf{This team may be better placed  in Observatory operations department }

\subsubsection{Science algorithms and pipelines}
This team is responsible to assess and assure the alert stream and annual data releases.
In the submitted proposal this includes extensive \gls{QA} to compare the data products against requirements -
this may be be better merged with System Performance/Verification.

The main responsibility  of this team  would then be the  underlying \gls{software} pipelines themselves.
That would include monitoring and updating the \gls{calibration} plan and algorithmic implementation. The Calibration Support Scientist on the Observatory Science team will be responsible for monitoring the physical implementation of the \gls{calibration} plan at the summit.
In \tabref{tab:FTE} this team is initially sized similarly to the AP/DRP teams in construction. There will be significant maintenance in the first two or three years of operations. As mentioned above there may be some consolidation with \gls{QA} activities in System Performance.

\subsubsection{Science platform  }
This team will be responsible for maintaining and evolving LSST’s user access portal, the \gls{Science Platform}. This will include keeping up with evolving technologies and computing infrastructure, as well as providing basic code-base maintenance, bug fixes, and low-level response to science community and internal \gls{LSST} requests for new features.

\subsubsection{Middleware }
In a service oriented model with a layered architecture as outlined in \secref{sec:arc} it is essential to have a cross cutting team who compose and debug services.
Software such as the \texttt{butler}  is not part of the \gls{pipeline} but the \gls{pipeline} needs it. In house developments such as \gls{Qserv} should be covered here (1.5FTE has been included for this those are  DOE/SLAC personnel, it could be 2FTE).
This would also cover the builds and how the code interacts with the infrastructure (\secref{sec:infra}.

\subsubsection{Infrastructure, Site Reliability Engineering  } \label{sec:infra}
This is for deployment of of various systems and pipelines. Configuration is included in this. There needs to be a couple of people who manage keys/secrets
for access to commodity services. We would need a security resource as well as database expertise.  This then implies using tooling for system management as
provided by e.g. \gls{AWS} console.

In general an \gls{SRE} team is responsible for the availability, latency, performance, efficiency, change management, monitoring, emergency response and capacity planning of their services \cite{Beyer:2016:SRE:3006357}.

This team would include paying for a liaison at any service provider e.g. Google Professional Services or a Service Manager at \gls{NCSA}. (2FTE calculated)

\subsubsection{Verification/Operations }
This team will take and verify new releases for operations before they are deployed to the operations system. They will monitor the operational system to make sure it is functioning - they should have some science knowledge to know it is actually working properly as opposed to not just giving errors. A team of 4 should be able to handle this.
Some support for this is assumed from IN2P3 .




